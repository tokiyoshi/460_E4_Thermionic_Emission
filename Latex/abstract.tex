The Richard-Dushman equation described the emission of electrons from a hot solid. With the addition of Schottky emission, we can compute the work function of a metal based on the relationship between temperature and the electron emission from a filament. These equations are derived from the statistical distributions of electrons in a solid and applying some basic assumptions about the electro-magnetic fields acting on such electrons. Using a circuit setup with a heated filament emitting electrons, along with a coaxial cylindrical anode measuring the emitted current, we explore the relationship between temperature of the wire and the anode current for the purpose of verifying the Richardson-Dushman equation. Using measured values we calculate the work function for the tungsten filament which is calculated to be 5.24 $\pm$ .54 eV as compared to an accepted value of 4.5 eV.