The purpose of this experiment was to take a look at thermonic emission, which explains a fundamental statistical/quantum property of metals that occurs in our daily lives through X-Rays and incandecent light bulbs. The emission of electrons with applied current is a relatively simple property to test, and finding their emission energies can be completed by using the phenomena of Schottky emission. With these details we can find the work function of a metal.

\par

In this lab we first investigate Schottky emission and validate that it is occurring for this field setup. We are able to find that our observations follow the theory of Schottky emission for all but one data point in our trials. It is unclear as to why this data point is deviating from theory.

\par

The next investigation is into Richard-Dushman equation, with the goal of finding the work function. We are able to find that the phenomena is well modelled by the theory and we are able to find a work function of 5.24 $\pm$ .54 eV which deviates 14\% from the accepted value of 4.5eV. This deviation which is not covered by our error bounds is likely due to a lack of accounting for the errors which occur in temperature estimates.