Thermionic emission is the outward flow of charged particles from a solid due to thermal excitation. First observed in the mid-1850's, the phenomena was investigated by a handful of scientists over then next half-century, including Thomas Edison who was interested in why parts of the filament in his light bulb darkened quicker than others. Thermionic emission is a common source of electrons in various applications; such as cathode ray tube, X-ray imaging and radio [].\\

Thermionic emission is the emission of electrons (or ions) from a hot metal or semiconductor, and can be considered an `evaporation' process. In analogy to an evaporating liquid, say water, thermionic emission is similar to the random escape of water molecules at temperatures below boiling. \\

Solids, such as metals or semiconductors, have an energy barrier for electrons leaving the surface. This energy barrier is referred to as the work function, and is dependent on the metal or semiconductor in question. Electrons with sufficient energy and momenta can exit the surface. An applied voltage across this solid serves to \textit{reduce} the effective work function of the solid -- promoting greater emission of electrons from the surface. This phenomena is known as Schottky emission and allows for a higher flux of electrons out of the surface.\\

Thermionic emission is described by the Richardson-Dushman equation which was, which was first described by Owen Richardson in 1901. The equation for the current of electrons emitted from a hot object is of the form,
\begin{align}
    J = A_0 T^2 e^{-w/kT}
\end{align}

In this paper we outline the theory behind the Richardson-Dushman equation and Schottky emission, outline a procedure for measuring the work function of a metal based on the relationship between temperature and emitted current, and finally compute the work function for a filament wire and explore the relationship between temperature and current.\\